\chapter{Descripción del sistema y requisitos}

\section{Descripción del sistema}

La idea general del sistema es ofrecer una forma automatizada de hacer los horarios de una escuela o facultad en base a restricciones tales como el profesorado y su disponibilidad, las aulas disponibles teniendo en cuenta su capacidad y el equipo del que disponen, el número de alumnos de una asignatura, bloques horarios, etc. 

Con este sistema podrán estudiarse distintas propuestas de horarios para maximizar el uso de las aulas y el rendimiento de todo el equipo de una facultad. 

% \section{Objetivos principales del sistema}

% \begin{enumerate}[OBJ-1]
%     \item Poder conectarse a las bases de datos de la universidad de forma que el usuario introduzca la menor información posible.
%     \item Poder crear un horario válido desde cero usando los datos disponibles.
%     \item Una vez hecha una propuesta de horario, realizar modificaciones sobre la misma de forma que se obtenga un horario válido.
% \end{enumerate}
