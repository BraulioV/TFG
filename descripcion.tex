\chapter{Descripción del sistema y requisitos}

\section{Descripción del sistema}

La idea general del sistema es ofrecer una forma automatizada de hacer los horarios de una escuela o facultad en base a restricciones tales como el profesorado y su disponibilidad, las aulas disponibles teniendo en cuenta su capacidad y el equipo del que disponen, el número de alumnos de una asignatura, bloques horarios, etc. 

Con este sistema podrán estudiarse distintas propuestas de horarios para maximizar el uso de las aulas, el rendimiento de todo el equipo de una facultad y facilitar el trabajo que supone la creación de un horario para cada nuevo curso.

\section{Objetivos principales del sistema}

\begin{enumerate}[OBJ-1]
    \item El usuario debe introducir la menor información posible para que le resulte más cómodo y sencillo hacer uso del sistema. Esto sería posible si la mayoría de información necesaria para la generación del horario fuera posible obtenerla de la base de datos de la escuela o de la UGR, para evitar que el usuario tenga que realizar ninguna entrada desde ficheros o similares.
    
    \item Tener la posibilidad de realizar un nuevo horario desde cero a partir de los datos disponibles, y la posibilidad de, a partir de uno ya generado, generar uno nuevo a partir de modificaciones.

    \item Sobre la solución del sistema, existen dos opciones:
    \begin{enumerate}[a)]
        \item Ofrecer una única solución y ofrecer la posibilidad de realizar modificaciones sobre esta de forma interactiva y que el sistema avise de posibles conflictos.
        \item Ofrecer como salida un parapeto de soluciones que haya encontrado el sistema y que el usuario final elija entre las soluciones que más le interesen.
    \end{enumerate}
    % \item Poder conectarse a las bases de datos de la universidad de forma que el usuario introduzca la menor información posible.
    % \item Poder crear un horario válido desde cero usando los datos disponibles.
    % \item Una vez hecha una propuesta de horario, realizar modificaciones sobre la misma de forma que se obtenga un horario válido.
\end{enumerate}
