\chapter{Descripción del sistema y requisitos}

\section{Descripción del sistema}

La idea general del sistema es ofrecer una forma automatizada de hacer los horarios de una escuela o facultad en base a restricciones tales como el profesorado y su disponibilidad, las aulas disponibles teniendo en cuenta su capacidad y el equipo del que disponen, el número de alumnos de una asignatura, bloques horarios, etc. 

Con este sistema podrán estudiarse distintas propuestas de horarios para maximizar el uso de las aulas, el rendimiento de todo el equipo de una facultad y facilitar el trabajo que supone la creación de un horario para cada nuevo curso.

\section{Objetivos principales del sistema}

\begin{enumerate}[OBJ-1]
    \item El usuario debe introducir la menor información posible para que le resulte más cómodo y sencillo hacer uso del sistema. Esto sería posible si la mayoría de información necesaria para la generación del horario fuera posible obtenerla de la base de datos de la escuela o de la UGR, para evitar que el usuario tenga que realizar ninguna entrada desde ficheros o similares.
    
    \item Tener la posibilidad de realizar un nuevo horario desde cero a partir de los datos disponibles, y la posibilidad de, a partir de uno ya generado, generar uno nuevo a partir de modificaciones.

    \item Sobre la solución del sistema, existen dos opciones:
    \begin{enumerate}[a)]
        \item Ofrecer una única solución y ofrecer la posibilidad de realizar modificaciones sobre esta de forma interactiva y que el sistema avise de posibles conflictos.
        \item Ofrecer como salida un parapeto de soluciones que haya encontrado el sistema y que el usuario final elija entre las soluciones que más le interesen.
    \end{enumerate}
    % \item Poder conectarse a las bases de datos de la universidad de forma que el usuario introduzca la menor información posible.
    % \item Poder crear un horario válido desde cero usando los datos disponibles.
    % \item Una vez hecha una propuesta de horario, realizar modificaciones sobre la misma de forma que se obtenga un horario válido.
\end{enumerate}

\section{Requisitos del sistema}

\begin{enumerate}[REQ-1]
    \item No pueden solaparse dos asignaturas el mismo día y a la misma hora en el mismo aula.
    \item Cómo máximo hay tres subgrupos de prácticas. Puede haber asignaturas en las que haya dos subgrupos y otras en las que haya sólo uno.
    \item Para cada grupo se debe de decidir de forma manual su franja horaria y el aula de teoría.
    \item No puede haber más de tres grupos de teoría asignados al mismo aula en el mismo turno (mañana o tarde). 
    \item Se debe saber de antemano el número de horas de teoría y prácticas de cada asignatura.
    \item Si el número de horas de teoría o prácticas de una asignatura es impar, se agrupará con otra asignatura.
    \item Para cada asignatura, dar la posibilidad de dar las horas de teoría seguidas o en días distintos. Debe haber al menos dos asignaturas que quieran dar sus horas de teoría en días separados para que pueda hacerse ésto.
    \item El horario obtenido para cada grupo no debe de contener huecos, es decir, debe ser lo más compacto posible.
    \item Las distintas especialidades tienen su horario en paralelo para no favorecer ninguna especialidad sobre otra.
    \item Se debe registrar el equipamiento disponible en cada laboratorio y el equipamiento que cada asignatura necesita para llevar a cabo sus prácticas para que el sistema realice la asignación a laboratorios de forma automática.
    \item Se debe poder elegir el número de días y la franja horaria para cada titulación por separado. 
\end{enumerate}
