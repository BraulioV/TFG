\chapter*{}
\thispagestyle{empty}
\cleardoublepage

\thispagestyle{empty}

% \input{portada/portada_2}



\cleardoublepage
\thispagestyle{empty}

\begin{center}
{\large\bfseries Generación automática de horarios: interfaz de usuario}\\
\end{center}
\begin{center}
Marta Gómez Macías (alumno)\\
\end{center}

%\vspace{0.7cm}
\noindent{\textbf{Palabras clave}: django, python, interfaz de usuario, web, usabilidad, seguridad, ingeniería del software}\\

\vspace{0.7cm}
\noindent{\textbf{Resumen}}\\

En este proyecto se ha realizado el diseño de la interfaz de usuario para un sistema de realización de horarios. Incluyendo aspectos como seguridad y la usabilidad.
\cleardoublepage


\thispagestyle{empty}


\begin{center}
{\large\bfseries Automatic generation of timetables: user interface designing}\\
\end{center}
\begin{center}
Marta Gómez Macías (student)\\
\end{center}

%\vspace{0.7cm}
\noindent{\textbf{Keywords}: django, python, user interface, web, usability, security, software engineering}\\

\vspace{0.7cm}
\noindent{\textbf{Abstract}}\\

In this project, an user interface for a timetabling system has been designed and implemented. Including aspects such as security and usability.

\chapter*{}
\thispagestyle{empty}

\noindent\rule[-1ex]{\textwidth}{2pt}\\[4.5ex]

Yo, \textbf{Marta Gómez Macías}, alumno de la titulación Grado en Ingeniería Informática de la \textbf{Escuela Técnica Superior
de Ingenierías Informática y de Telecomunicación de la Universidad de Granada}, con DNI 75929776Z, autorizo la
ubicación de la siguiente copia de mi Trabajo Fin de Grado en la biblioteca del centro para que pueda ser
consultada por las personas que lo deseen.

\vspace{6cm}

\noindent Fdo: Marta Gómez Macías

\vspace{2cm}

\begin{flushright}
Granada a 11 de Septiembre de 2017.
\end{flushright}


\chapter*{}
\thispagestyle{empty}

\noindent\rule[-1ex]{\textwidth}{2pt}\\[4.5ex]

D. \textbf{Joaquín Fernández Valdivia (tutor1)}, Profesor del Área de Algorítmica del Departamento DECSAI de la Universidad de Granada.

\vspace{0.5cm}

D. \textbf{Jose Antonio García Soria (tutor2)}, Profesor del Área de Algorítmica del Departamento DECSAI de la Universidad de Granada.


\vspace{0.5cm}

\textbf{Informan:}

\vspace{0.5cm}

Que el presente trabajo, titulado \textit{\textbf{Generación automática de horarios: interfaz de usuario}},
ha sido realizado bajo su supervisión por \textbf{Marta Gómez Macías (alumno)}, y autorizamos la defensa de dicho trabajo ante el tribunal
que corresponda.

\vspace{0.5cm}

Y para que conste, expiden y firman el presente informe en Granada a 11 de Septiembre de 2017 .

\vspace{1cm}

\textbf{Los directores:}

\vspace{5cm}

\noindent \textbf{Joaquín Fernández Valdivia (tutor1) \ \ \ \ \ Jose Antonio García Soria (tutor2)}

\chapter*{Agradecimientos}
\thispagestyle{empty}

       \vspace{1cm}


Quiero agradecer a \textbf{Braulio Vargas López}, mi compañero en este proyecto y en mil más, por apoyarme y animarme cada día. Espero poder hacer muchos más proyectos juntos y vivir un montón de aventuras los dos.

Quiero agradecer también a todos los profesores que me han enseñado todo lo que sé a lo largo de la carrera, en especial quiero mencionar a los siguientes:

\begin{enumerate}[---]
	\item A \textbf{Francisco Miguel García Olmedo}, mi profesor de álgebra y lambda calculus, pero que también me ha enseñado \LaTeX, Python, Haskell y mil cosas más. Se nota que te encanta enseñar y te apasiona tu trabajo. Mil gracias por todo lo que me has enseñado.
	\item A \textbf{Jesús García Miranda}, por su inestimable ayuda cuando cursé LMD y también en este proyecto. Sin ti no lo hubiéramos conseguido.
	\item A \textbf{Rosa María Rodríguez Sánchez}, mi profesora de Estructuras de Datos. Contigo aprendí todos los trucos escondidos de C++. Siempre has sido un modelo a seguir para mí, ¡desde el primer día de clase! Me has inspirado a seguir adelante.
	\item Y a \textbf{Francisco José Cortijo Bon}, mi primer profesor de programación. Gracias a las buenas prácticas que aprendí contigo he conseguido destacar frente a otros programadores.
\end{enumerate}

Por último, y no menos importante, me gustaría agradecer a 

\begin{enumerate}[---]
	\item \textbf{Alberto Rodríguez Frías} por haberme animado a hacer las pruebas de selección de \textit{Wazuh}. De no ser por ti, ¡a saber dónde estaría ahora! Mil gracias por haberme escrito, a pesar de no conocernos de nada.
	\item \textbf{Jesús Linares Bolaños} por haberme enseñado tanto sobre seguridad en servidores, programación... Aprecio muchísimo cada rato que echamos juntos revisando mi trabajo.
	\item \textbf{Víctor Manuel Fernández Castro} por tu paciencia infita y tus consejos. Aún me queda un montón que aprender de ti.
\end{enumerate}
