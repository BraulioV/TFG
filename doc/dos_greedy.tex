\chapter{Segunda aproximación Greedy}
Para resolver los problemas de la primera aproximación realizada, se ha desarrollado un segundo algoritmo greedy. A esta segunda aproximación se le ha añadido un algoritmo previo que sirve para hacer la estructura del horario y cuyo objetivo es minimizar la ocupación de los laboratorios de prácticas.

\section{Preasignación de horas}
En esta fase lo único que se decide es qué horas serán de prácticas y cuáles serán de teoría sin asignar ninguna asignatura en concreto. El objetivo de esta fase de preasignación es minimizar la ocupación de los laboratorios de prácticas y evitar así que se saturen.

Por sencillez, el algoritmo recibe como parámetro una franja horaria (mañana o tarde) por lo que antes de empezar debe filtrar los grupos que pertenecen a dicha franja horaria. Una vez filtrados empieza a iterar sobre todos los cursos. 

En primer lugar, calcula las horas de teoría y de prácticas que tiene cada curso y, después, calcula el número total de horas de prácticas que hay en los laboratorios. Por ejemplo, si los laboratorios están abiertos cinco días a la semana y cuatro horas al día tendríamos un total de 20 horas. Como cada turno de prácticas tiene dos horas, nos quedarían un total de diez turnos de prácticas por asignar. Si, por ejemplo, tuviésemos que asignar doce turnos de prácticas, tendríamos que elegir de forma aleatoria dos turnos que se repetirían. Debe ser de forma aleatoria para que los turnos ``repetidos'' no coincidan en los distintos cursos y se saturen los laboratorios un día en concreto.

Después, el algoritmo empieza a iterar sobre cada grupo de ese curso y va asignando horas de prácticas y teoría en los huecos ``libres''. Para saber qué huecos están libres se usa una matriz auxiliar que registra qué horas son de prácticas en cada curso.

\subsection{Pseudocódigo del algoritmo}
El siguiente pseudocódigo refleja de forma más clara el funcionamiento del algoritmo.

\newpage
\begin{pseudocode}{GreedyEstructura}{turno}
\label{greedyestructura}
filtro\_grupos \GETS \CALL{FiltrarGrupos}{turno}\\
principio \GETS \CALL{ComienzoFranja}{turno}\\
final \GETS \CALL{FinalFranja}{turno}\\
\FOREACH c \in cursos \DO
\BEGIN
    filtro\_cursos \GETS \CALL{FiltrarGruposCurso}{curso, filtro\_grupos}\\
    h\_lab \GETS \CALL{CalcularHorasPracticasCurso}{curso}\\
    h\_week \GETS dias\_semana * n\_turnos\\

    \IF h\_lab > h\_week \THEN
	    	dias\_rep \GETS \CALL{CalcularTurnosRepetidos}{ }\\
    \ELSE 
      dias\_rep \GETS \emptyset\\

   	tabla\_lab \GETS \CALL{Matriz}{dias\_semana, n\_turnos}\\

   	\FOREACH g \in filtro\_cursos \DO
   	\BEGIN
   		teoria \GETS \CALL{CalcularHorasTeoria}{g}\\
   		lab \GETS \CALL{CalcularHorasLab}{g}\\

   		\FOR hora \GETS 0 \TO final; i+=2 \DO
   		\BEGIN
   		   \FOR dia \in SEMANA \DO
           \BEGIN
                \IF \NOT tabla\_lab[hora, dia] \AND lab \ge 2 \THEN
                \BEGIN
                    \CALL{AsignarEstructura}{hora, dia, L}\\
                    tabla\_lab[hora, dia] = True\\
                    tabla\_lab[hora + 1, dia] = True\\
                    lab \GETS lab - 2\\
                \END
                \ELSEIF tabla\_lab[hora, dia] \AND lab \ge 2 \AND (dia, hora) \in dias\_rep \THEN
                \BEGIN
                    \CALL{delete}{dias\_rep(dia, hora)}\\
                    \CALL{AsignarEstructura}{hora, dia, L}\\
                    lab \GETS lab - 2\\
                \END
                \ELSEIF tabla\_lab[hora, dia] \AND teoria \ge 2 \THEN
                \BEGIN
                    \CALL{AsignarEstructura}{hora, dia, T}\\
                    teoria \GETS teoria - 2\\
                \END
                \ELSEIF teoria \ge 2 \THEN
                \BEGIN
                    \CALL{AsignarEstructura}{hora, dia, T}\\
                    teoria \GETS teoria - 2\\
                \END
                \ELSEIF lab \ge 2 \THEN
                \BEGIN
                    \CALL{AsignarEstructura}{hora, dia, L}\\
                    lab \GETS lab - 2\\
                \END
                \ELSEIF teoria = 1 \THEN
                \BEGIN
                  \CALL{AsignarEstructura}{hora, dia, T}\\
                  teoria \GETS teoria - 1\\
                \END
                \ELSEIF lab = 1 \THEN
                \BEGIN
                  \CALL{AsignarEstructura}{hora, dia, L}\\
                  lab \GETS lab - 1\\
                \END
                \ELSE \CALL{break}{}
           \END
   		\END
   	\END
\END
\end{pseudocode}

\section{Asignación de horas de teoría}
Esta segunda versión del algoritmo de asignación de horas de teoría utiliza como base la estructura hecha por el algoritmo de preasignación. Así, sólo tiene permitido asignar horas en los huecos que estén marcados como \texttt{T} en la matriz de estructura. Usando esto como base y un diccionario con las horas por asignar de cada asignatura, el algoritmo distingue varios casos diferentes:

\begin{enumerate}[---]
  \item Si la hora actual no se corresponde a una de teoría en la matriz de estructura, el algoritmo no hace nada.
  \item Si la hora actual sí es de teoría y además la asignatura a asignar tiene más de dos horas pendientes por asignar, se crea un bloque de dos horas con dicha asignatura.
  \item Si la hora actual sí es de teoría pero la asignatura a asignar sólo tiene una hora más por asignar, el algoritmo trata de buscar otra con la que hacer un bloque de dos. En el caso de no existir, se asignaría esa hora únicamente.
\end{enumerate}

Cuando a una asignatura no le quedan más horas por asignar, se elimina automáticamente de la lista de asignaturas por asignar. El algoritmo pasa al siguiente grupo cuando dicha lista se queda vacía.

\subsection{Comparación con la primera versión}
En comparación con la primera versión, hay bastantes cosas que hemos mejorado:

\begin{enumerate}[$\bullet$]
  \item En la versión anterior, si no se conseguía hacer un bloque de dos con dos horas sueltas, el algoritmo o bien ciclaba o bien dejaba huecos en el horario, dando lugar a una estructura no deseada.
  \item En consecuencia, el algoritmo es más rápido que el anterior desarrollado.
  \item Usando el algoritmo de preasignación, hemos conseguido que el código sea mucho más sencillo y legible.
  \item También, el algoritmo ha perdido esa alta diversidad que tenía la versión original puesto que en la nueva versión se ha acotado mucho la estructura del algoritmo y el espacio de búsqueda.
\end{enumerate}

En conclusión, aunque el algoritmo haya perdido su diversidad original, hemos ganado tanto en eficiencia como en calidad.

\subsection{Pseudocódigo del algoritmo}
El siguiente pseudocódigo refleja de forma más clara el funcionamiento del algoritmo.

\begin{pseudocode}{GreedyTeoria}{ }
\label{greedyteoria}
\FOREACH g \in grupos
\BEGIN
  subject\_list \GETS \CALL{ObtenerListaAsignaturas}{g}\\
  \CALL{Barajar}{subject\_list}\\
  horas \GETS \CALL{DiccionarioHoras}{subject\_list}\\
  s \GETS 0\\

  \IF g.turno = M \THEN
  \BEGIN
    empezar \GETS 0 \\
    acabar \GETS \frac{horas\_dia}{2}\\
  \END
  \ELSE 
  \BEGIN
    empezar \GETS \frac{horas\_dia}{2}\\
    acabar \GETS horas\_dia\\
  \END\\

  \FOR h \GETS empezar \TO acabar; h+=2\\
  \BEGIN
    \FOR d \GETS 0 \TO W_d
    \BEGIN
      \IF subject\_list = \emptyset \THEN
      \BREAK\\

      \ELSEIF \NOT (\CALL{EsTeoria}{g, h, d} \AND \\ \;\;\; \CALL{EsTeoria}{g,h+1,d}) \THEN
      \BEGIN
        remove \GETS \FALSE\\
        asignado \GETS \FALSE\\
      \END\\

      \ELSEIF \CALL{EsTeoria}{g, h, d} \AND \\ \;\;\; horas[subject\_list[s]] \geq 2 \THEN
      \BEGIN
        eliminar \GETS \CALL{AsignarCeldaT}{g,h,d,2}\\
        asignado \GETS \TRUE\\
      \END\\

      \ELSEIF \CALL{EsTeoria}{g, h, d} \AND \\ \;\;\; horas[subject\_list[s]] = 1 \THEN
      \BEGIN
        eliminar \GETS \CALL{AsignarCeldaT}{g,h,d,1}\\
        asignado \GETS \TRUE\\
      \END\\
      \ELSEIF \CALL{EsTeoria}{g, h+1, d} \AND \\ \;\;\; horas[subject\_list[s]] = 1 \THEN
      \BEGIN
        eliminar \GETS \CALL{AsignarCeldaT}{g,h+1,d,1}\\
        asignado \GETS \TRUE\\
      \END\\

      \IF eliminar \THEN \CALL{EliminarAsignaturasCero}{ }\\
      \ELSE s = s+1 \pmod{subject\_list.size()}\\
    \END
  \END 
\END
\end{pseudocode}

\section{Asignación de horas de prácticas}

\subsection{Análisis de la versión anterior}

Esta nueva versión del algoritmo voraz para asignar las horas de prácticas y aulas a los distintos grupos, sigue la misma filosofía de la utilización de la ``\textit{ventana}'' para asignar los grupos. 

En la versión anterior, lo que se hacía era, partiendo de una permutación inicial de las distintas asignaturas que tenía el grupo en un cuatrimiestre dado, se generaba una nueva permutación en la que los elementos de esta eran grupos de tres asignaturas o tantas como subgrupos de prácticas haya, que se correspondían con la asignación de las asignaturas, para cada día. A continuación, podemos ver un ejemplo de lo que realizaba esta ventana:
\begin{table}[H]
\begin{center}
\begin{tabular}{|c|c|c|c|c|}
\hline
FFT & CA & ALEM & FS & FP\\
\hline
\end{tabular}
\caption{Permutación inicial.}
\end{center}
\end{table}
Al haber en primero tres subgrupos de prácticas, tenemos lo siguiente:
\begin{table}[H]
\begin{center}
\begin{tabular}{|c|c|c|c|}
\hline
(FFT, CA, ALEM) & (CA, ALEM, FS) & (ALEM, FS, FP) & $\cdots$ \\
\hline
\end{tabular}
\caption{Permutaciones generadas por la ventana.}
\end{center}
\end{table}

Con esto, podríamos ver cómo el primer día de prácticas correspondería a Fundamentos Físicos y Tecnológicos para el grupo 1, Cálculo para el grupo 2 y Álgebra Lineal y Estructuras Matemáticas para el grupo 3.

Esto hace que asignar las horas de prácticas a los distintos días de la semana sea muy fácil, ya que, a partir de una permutación inicial, vamos moviendo una ventana que toma $n$ asignaturas, y se va desplazando una posición a la derecha para cada día de la semana. 

El problema que tiene el realizar la asignación de esta forma, como se hacía en la primera versión, es que si todas las asignaturas del plan de estudio tuvieran la misma cantidad de horas de prácticas y teoría, esta asignación funcionaría sin problemas ninguno. Esto sería el caso ideal, pero la realidad no es así, sino que en un mismo año del plan de estudios, para un semestre dado, podemos encontrarnos asignaturas con una, dos o incluso tres horas de prácticas en un mismo grupo, por lo que este algoritmo, en esta versión tan simple, no funciona. 

Además de esto, el consumo de memoria de esta versión del algoritmo es muy elevado, ya que a partir de la permutación inicial, generar una nueva lista con las ternas de asignaturas, supone repetir varias veces el mismo elemento, lo que supone un mal gasto de memoria innecesario, que podría hacer que máquinas con menor potencia de cálculo y/o recursos, se vean afectadas en su rendimiento.

\subsection{Formulación de la siguiente versión}

Teniendo estos problemas en mente de la versión anterior, procedimos a realizar una nueva versión del algoritmo de prácticas. Esta nueva versión parte desde un punto más ventajoso que la versión original, gracias a la preasignación de celdas, el algoritmo solo tiene que ir mirando qué celdas de la estructura del horario son para prácticas y cuáles no, y asignar las asignaturas donde corresponda.

La elección de qué asignaturas van a cada día, se realizará con la misma idea que la versión anterior, utilizando la ventana deslizante. Pero en este caso, la idea se ha refinado mucho más que la versión anterior. 

La diferencia más básica es que ya no se genera una lista con ternas de asignaturas, sino lo que se mantiene es una lista de índices, que marcan qué asignatura va para cada grupo, y se va sumando uno módulo número de asignaturas del grupo, para ese cuatrimestre. Esto reduce la memoria y simplifica mucho el código.
